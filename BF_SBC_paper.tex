% Options for packages loaded elsewhere
\PassOptionsToPackage{unicode}{hyperref}
\PassOptionsToPackage{hyphens}{url}
\PassOptionsToPackage{dvipsnames,svgnames,x11names}{xcolor}
%
\documentclass[
  letterpaper,
  DIV=11,
  numbers=noendperiod]{scrartcl}

\usepackage{amsmath,amssymb}
\usepackage{iftex}
\ifPDFTeX
  \usepackage[T1]{fontenc}
  \usepackage[utf8]{inputenc}
  \usepackage{textcomp} % provide euro and other symbols
\else % if luatex or xetex
  \usepackage{unicode-math}
  \defaultfontfeatures{Scale=MatchLowercase}
  \defaultfontfeatures[\rmfamily]{Ligatures=TeX,Scale=1}
\fi
\usepackage{lmodern}
\ifPDFTeX\else  
    % xetex/luatex font selection
\fi
% Use upquote if available, for straight quotes in verbatim environments
\IfFileExists{upquote.sty}{\usepackage{upquote}}{}
\IfFileExists{microtype.sty}{% use microtype if available
  \usepackage[]{microtype}
  \UseMicrotypeSet[protrusion]{basicmath} % disable protrusion for tt fonts
}{}
\makeatletter
\@ifundefined{KOMAClassName}{% if non-KOMA class
  \IfFileExists{parskip.sty}{%
    \usepackage{parskip}
  }{% else
    \setlength{\parindent}{0pt}
    \setlength{\parskip}{6pt plus 2pt minus 1pt}}
}{% if KOMA class
  \KOMAoptions{parskip=half}}
\makeatother
\usepackage{xcolor}
\setlength{\emergencystretch}{3em} % prevent overfull lines
\setcounter{secnumdepth}{-\maxdimen} % remove section numbering
% Make \paragraph and \subparagraph free-standing
\makeatletter
\ifx\paragraph\undefined\else
  \let\oldparagraph\paragraph
  \renewcommand{\paragraph}{
    \@ifstar
      \xxxParagraphStar
      \xxxParagraphNoStar
  }
  \newcommand{\xxxParagraphStar}[1]{\oldparagraph*{#1}\mbox{}}
  \newcommand{\xxxParagraphNoStar}[1]{\oldparagraph{#1}\mbox{}}
\fi
\ifx\subparagraph\undefined\else
  \let\oldsubparagraph\subparagraph
  \renewcommand{\subparagraph}{
    \@ifstar
      \xxxSubParagraphStar
      \xxxSubParagraphNoStar
  }
  \newcommand{\xxxSubParagraphStar}[1]{\oldsubparagraph*{#1}\mbox{}}
  \newcommand{\xxxSubParagraphNoStar}[1]{\oldsubparagraph{#1}\mbox{}}
\fi
\makeatother


\providecommand{\tightlist}{%
  \setlength{\itemsep}{0pt}\setlength{\parskip}{0pt}}\usepackage{longtable,booktabs,array}
\usepackage{calc} % for calculating minipage widths
% Correct order of tables after \paragraph or \subparagraph
\usepackage{etoolbox}
\makeatletter
\patchcmd\longtable{\par}{\if@noskipsec\mbox{}\fi\par}{}{}
\makeatother
% Allow footnotes in longtable head/foot
\IfFileExists{footnotehyper.sty}{\usepackage{footnotehyper}}{\usepackage{footnote}}
\makesavenoteenv{longtable}
\usepackage{graphicx}
\makeatletter
\def\maxwidth{\ifdim\Gin@nat@width>\linewidth\linewidth\else\Gin@nat@width\fi}
\def\maxheight{\ifdim\Gin@nat@height>\textheight\textheight\else\Gin@nat@height\fi}
\makeatother
% Scale images if necessary, so that they will not overflow the page
% margins by default, and it is still possible to overwrite the defaults
% using explicit options in \includegraphics[width, height, ...]{}
\setkeys{Gin}{width=\maxwidth,height=\maxheight,keepaspectratio}
% Set default figure placement to htbp
\makeatletter
\def\fps@figure{htbp}
\makeatother

\KOMAoption{captions}{tableheading}
\makeatletter
\@ifpackageloaded{caption}{}{\usepackage{caption}}
\AtBeginDocument{%
\ifdefined\contentsname
  \renewcommand*\contentsname{Table of contents}
\else
  \newcommand\contentsname{Table of contents}
\fi
\ifdefined\listfigurename
  \renewcommand*\listfigurename{List of Figures}
\else
  \newcommand\listfigurename{List of Figures}
\fi
\ifdefined\listtablename
  \renewcommand*\listtablename{List of Tables}
\else
  \newcommand\listtablename{List of Tables}
\fi
\ifdefined\figurename
  \renewcommand*\figurename{Figure}
\else
  \newcommand\figurename{Figure}
\fi
\ifdefined\tablename
  \renewcommand*\tablename{Table}
\else
  \newcommand\tablename{Table}
\fi
}
\@ifpackageloaded{float}{}{\usepackage{float}}
\floatstyle{ruled}
\@ifundefined{c@chapter}{\newfloat{codelisting}{h}{lop}}{\newfloat{codelisting}{h}{lop}[chapter]}
\floatname{codelisting}{Listing}
\newcommand*\listoflistings{\listof{codelisting}{List of Listings}}
\makeatother
\makeatletter
\makeatother
\makeatletter
\@ifpackageloaded{caption}{}{\usepackage{caption}}
\@ifpackageloaded{subcaption}{}{\usepackage{subcaption}}
\makeatother

\ifLuaTeX
  \usepackage{selnolig}  % disable illegal ligatures
\fi
\usepackage{bookmark}

\IfFileExists{xurl.sty}{\usepackage{xurl}}{} % add URL line breaks if available
\urlstyle{same} % disable monospaced font for URLs
\hypersetup{
  pdftitle={Simulation-based calibration checking for Bayes Factors},
  pdfauthor={Martin Modrák, Paul Bürkner},
  colorlinks=true,
  linkcolor={blue},
  filecolor={Maroon},
  citecolor={Blue},
  urlcolor={Blue},
  pdfcreator={LaTeX via pandoc}}


\title{Simulation-based calibration checking for Bayes Factors}
\author{Martin Modrák, Paul Bürkner}
\date{}

\begin{document}
\maketitle


This document reproduces all figures and numbers for the relevant
sections.

\subsection{Toy examples}\label{toy-examples}

\subsubsection{Single binary
observation}\label{single-binary-observation}

The various test statistics have the expected random behavior and only
rarely drop below the relevant thresholds
(Figure~\ref{fig-binary-correct}).

\begin{figure}

\centering{

\includegraphics{BF_SBC_paper_files/figure-pdf/fig-binary-correct-1.pdf}

}

\caption{\label{fig-binary-correct}Histories of check statistics for the
correct binary model. The horizontal blue line marks rejecting the
hypothesis of calibration with 5\% false positive rate. We see that in
all cases, violations tend to be non-severe and short-lived A) shows the
log gamma statistic of default SBC, B) is the p-value of the
bootstrapped miscalabration test of Dimitraidis et al.~and C) is the
p-value from t-test for the data-averaged posterior.}

\end{figure}%

\begin{figure}

\centering{

\includegraphics{BF_SBC_paper_files/figure-pdf/fig-binary-flipped-1.pdf}

}

\caption{\label{fig-binary-flipped}Histories of check statistics for the
flipped binary model. The horizontal blue line marks rejecting the
hypothesis of calibration with 5\% false positive rate. The vertical
orange line marks the number of simulations when the check first attains
80\% power. We see that SBC as well as miscalibration identify the
problem very quickly, while data-averaged posterior (DAP) does not
diagnose it (the rare low p-values arise from using a t-test on what is
essentially binary data). Note that the p-values from the miscalibration
test are capped at \(\frac{1}{2000}\) due to the number of bootstrap
samples used.}

\end{figure}%

\subsubsection{Poisson vs.~negative
binomial}\label{poisson-vs.-negative-binomial}

\begin{figure}

\centering{

\includegraphics{BF_SBC_paper_files/figure-pdf/fig-pnb-ignore-all-1.pdf}

}

\caption{\label{fig-pnb-ignore-all}Histories of SBC gamma statistic for
poisson-NB model ignoring all data in BF computation, the model index
and two data-dependent derived quantities: the log likelihood (log\_lik)
and variance estimate (var\_y).}

\end{figure}%

\begin{figure}

\centering{

\includegraphics{BF_SBC_paper_files/figure-pdf/fig-pnb-ignore-half-1.pdf}

}

\caption{\label{fig-pnb-ignore-half}Histories of SBC gamma statistics
for poisson-NB model ignoring half the data in BF computation.}

\end{figure}%

\begin{figure}

\centering{

\includegraphics{BF_SBC_paper_files/figure-pdf/fig-pnb-noise-1.pdf}

}

\caption{\label{fig-pnb-noise}Histories of check statistics for
poisson-NB model with noise in BF. A) shows the log gamma statistic of
default SBC, B) is the p-value of the bootstrapped miscalabration test
of Dimitraidis et al.~and C) is the p-value from t-test for the
data-averaged posterior.}

\end{figure}%

\begin{figure}

\centering{

\includegraphics{BF_SBC_paper_files/figure-pdf/fig-pnb-bias-1.pdf}

}

\caption{\label{fig-pnb-bias}Histories of check statistics for biased
poisson-NB model. A) shows the log gamma statistic of default SBC, B) is
the p-value of the bootstrapped miscalabration test of Dimitraidis et
al.~and C) is the p-value from t-test for the data-averaged posterior.}

\end{figure}%

\subsection{Realistic examples}\label{realistic-examples}

In this example, rejections change the implied prior to
\(P(\mathcal{M}_1) \simeq 0.6\) and after running 2 000 we find no
problems .

For the second check, \ldots{} but to show an alternative, we can also
choose \(P(M_1) \simeq 0.77\), which results in
\(P(M_1\mid y_1) = \frac{1}{2}\)

After running 2 000 simulations we see no calibration issues .

\subsubsection{Discovering bad normalization
constants}\label{discovering-bad-normalization-constants}

\begin{figure}

\centering{

\includegraphics{BF_SBC_paper_files/figure-pdf/fig-turtles-bad-norm-1.pdf}

}

\caption{\label{fig-turtles-bad-norm}Histories of check statistics for
bad normalization constant in the turtles model. A) shows the log gamma
statistic of default SBC, B) is the p-value of the bootstrapped
miscalabration test of Dimitraidis et al.~and C) is the p-value from
t-test for the data-averaged posterior.}

\end{figure}%

\subsubsection{Posterior SBC and the importance of using correct
priors}\label{posterior-sbc-and-the-importance-of-using-correct-priors}

Employing posterior SBC using a dataset with 4 groups and 3 observations
each as \(y_1\) (to make all parameters well identified for sampling)
produces a proper prior for the intercept and standard deviations and
results in perfect calibration in all methods even after running 10 000
simulations (Figure~\ref{fig-ranef-post}; ).

\begin{figure}

\centering{

\includegraphics{BF_SBC_paper_files/figure-pdf/fig-ranef-constant-1.pdf}

}

\caption{\label{fig-ranef-constant}Histories of check statistics for
random effect model using constant intercept and standard deviation.}

\end{figure}%

\begin{figure}

\centering{

\includegraphics{BF_SBC_paper_files/figure-pdf/fig-ranef-post-1.pdf}

}

\caption{\label{fig-ranef-post}Result of calibration checks for random
effect model using posterior SBC. A) empirical CDF difference plots for
model parameters (g\_Subject --- standard deviation of the random
intercepts, model --- model index, mu --- overall intercept, sig2 ---
residual variance, Subject --- merged plot for all random intercepts, x1
--- fixed effect) B) calibration plot.}

\end{figure}%

(for datasets with \(\text{sd}(y) \geq 1\) miscalibration p \textless{}
0.001, Figure~\ref{fig-ttest-fixed} B,C).

\begin{figure}

\centering{

\includegraphics{BF_SBC_paper_files/figure-pdf/fig-ttest-fixed-1.pdf}

}

\caption{\label{fig-ttest-fixed}Results of SBC when using a fixed value
of \(\sigma\) when simulating datasets for Bayesian t-test. A) Histories
of log gamma statistics for the model index, posterior mean (mu) and
\(f(i, y) = (i - \frac{1}{2})(\text{sd}(y) - 1)\) (model\_sd\_m1). We
see that only the latter diagnoses any problems. B) Calibration plot
across all simulations shows no problems. C) and D) Separate calibration
plots for datasets with high/low standard deviation show problems.}

\end{figure}%

When we use posterior SBC (with \(y_1 = (-1,1)\)), the computed Bayes
factors pass all checks to high precision --- we have run 50 000
simulations and see no sign of problems including all derived quantities
().

\subsection{Appendix}\label{appendix}

\subsubsection{Metrics to assess data-averaged
posterior}\label{metrics-to-assess-data-averaged-posterior}

\begin{figure}

\centering{

\includegraphics{BF_SBC_paper_files/figure-pdf/fig-dap-probs-1.pdf}

}

\caption{\label{fig-dap-probs}}

\end{figure}%

\begin{longtable}[]{@{}lllll@{}}
\toprule\noalign{}
N & scenario & T-test & Gaffke & Bayesian t-test \\
\midrule\noalign{}
\endhead
\bottomrule\noalign{}
\endlastfoot
10 & Good Cauchy & 20.80\% & 0.00\% & 14.20\% \\
& Good Normal & 6.40\% & 0.70\% & 0.10\% \\
& Ranef presence Post SBC & 4.20\% & 1.10\% & 0.30\% \\
20 & Good Cauchy & 10.60\% & 0.10\% & 5.30\% \\
& Good Normal & 5.50\% & 2.00\% & 0.20\% \\
& Ranef presence Post SBC & 5.50\% & 2.70\% & 0.10\% \\
50 & Good Cauchy & 7.00\% & 1.60\% & 1.30\% \\
& Good Normal & 5.30\% & 3.00\% & 0.20\% \\
& Ranef presence Post SBC & 5.50\% & 3.40\% & 0.20\% \\
100 & Good Cauchy & 5.10\% & 2.10\% & 0.70\% \\
& Good Normal & 4.90\% & 3.60\% & 0.00\% \\
& Ranef presence Post SBC & 4.90\% & 3.50\% & 0.00\% \\
\end{longtable}

\begin{longtable}[]{@{}lllll@{}}
\toprule\noalign{}
N & scenario & T-test & Gaffke & Bayesian t-test \\
\midrule\noalign{}
\endhead
\bottomrule\noalign{}
\endlastfoot
10 & Poisson NB - bias & 86.30\% & 59.70\% & 23.10\% \\
& Ranef presence fixed & 10.50\% & 4.00\% & 0.10\% \\
& Turtles - bad normalization & 14.40\% & 3.20\% & 2.80\% \\
20 & Poisson NB - bias & 99.70\% & 98.60\% & 80.60\% \\
& Ranef presence fixed & 14.20\% & 9.10\% & 0.40\% \\
& Turtles - bad normalization & 16.70\% & 6.60\% & 1.80\% \\
50 & Poisson NB - bias & 100.00\% & 100.00\% & 100.00\% \\
& Ranef presence fixed & 34.10\% & 28.90\% & 3.50\% \\
& Turtles - bad normalization & 28.20\% & 18.80\% & 2.70\% \\
100 & Poisson NB - bias & 100.00\% & 100.00\% & 100.00\% \\
& Ranef presence fixed & 65.80\% & 61.60\% & 14.10\% \\
& Turtles - bad normalization & 50.70\% & 42.00\% & 9.60\% \\
\end{longtable}

\begin{verbatim}
[1] 0.02649853
\end{verbatim}

\begin{verbatim}
[1] 0.03146846
\end{verbatim}

\subsubsection{Good check convergence}\label{good-check-convergence}

Empirically, we can see the lack of any convergence for the Cauchy case
and very slow convergence for the normal case in
Figure~\ref{fig-good-convergence}. It follows, that the Good check
cannot reliably diagnose BF computation unless we know that the BF
distribution is well behaved.

\begin{figure}

\centering{

\includegraphics{BF_SBC_paper_files/figure-pdf/fig-good-convergence-1.pdf}

}

\caption{\label{fig-good-convergence}Convergence of the good check when
a single datapoint is simulated from the standard normal and A)
\(\mathcal{M}_0: y \sim \text{Cauchy}(0, 1)\) --- here the variance is
infinite and we see no convergence at all or B)
\(\mathcal{M}_0: y \sim N(2, 1)\) where the average Bayes factor
eventually converges, but 50 000 simulations are not enough for reliable
convergence. Each line is the cumulative average from a single set of
simulations. The highlighted area shows average BF 0.9 - 1.1. Means from
first 1000 simulations are not shown.}

\end{figure}%

\paragraph{SBC for the same models}\label{sbc-for-the-same-models}

\begin{figure}

\centering{

\includegraphics{BF_SBC_paper_files/figure-pdf/fig-good-sbc-1.pdf}

}

\caption{\label{fig-good-sbc}ECDF difference plots from SBC (A) and
calibration diagrams (B, C) for Bayes factors when
\(\mathcal{M}_1: y \sim N(0, 1)\) and either
\(\mathcal{M}_0: y \sim \text{Cauchy}(0, 1)\) or
\(\mathcal{M}_0: y \sim N(2, 1)\).}

\end{figure}%

\begin{verbatim}
N sims Cauchy:  10000 
\end{verbatim}

\begin{verbatim}
95% CI for DAP difference from prior: -0.0032 -- 0.0038; miscalibration: 0.0008, 95% quantile under null: 0.001; SBC sensitive to eCDF difference up to 0.015
\end{verbatim}

\begin{verbatim}
N sims norm2:  10000 
\end{verbatim}

\begin{verbatim}
95% CI for DAP difference from prior: -0.0141 -- 0.0005; miscalibration: 0.0011, 95% quantile under null: 0.0013; SBC sensitive to eCDF difference up to 0.015
\end{verbatim}

\subsubsection{\texorpdfstring{Miscalibration with surrogate
distributions for variance in
\texttt{ttestBF}}{Miscalibration with surrogate distributions for variance in ttestBF}}\label{miscalibration-with-surrogate-distributions-for-variance-in-ttestbf}

First we instead use a proper prior with similar form
\(\pi_\text{prior}(\sigma^2) = \frac{1}{1 + \sigma^2}\). We require 20
000 simulations to detect mild miscalibration which is visible in the
specifically designed derived quantity as well as when splitting the
results based on the standard deviation of the data, see
Figure~\ref{fig-ttest-inv1psquared}. We see similarly mild problems when
instead using \(\sigma \sim \text{HalfCauchy}(0,1)\), see
Figure~\ref{fig-ttest-cauchy}.

\begin{figure}

\centering{

\includegraphics{BF_SBC_paper_files/figure-pdf/fig-ttest-inv1psquared-1.pdf}

}

\caption{\label{fig-ttest-inv1psquared}Results of SBC when using a
\(\pi_\text{prior}(\sigma^2) = \frac{1}{1 + \sigma^2}\) when simulating
datasets for Bayesian t-test. A) ECDF difference plot for the derived
quantity \(f(i, y) = (i - \frac{1}{2})(\text{sd}(y) - 1)\)
(model\_sd\_m1). B) and C) Separate binary prediction calibration plots
for datasets with high/low standard deviation.}

\end{figure}%

\begin{figure}

\centering{

\includegraphics{BF_SBC_paper_files/figure-pdf/fig-ttest-cauchy-1.pdf}

}

\caption{\label{fig-ttest-cauchy}Results of SBC when using a
\(\sigma \sim \text{HalfCauchy}(0,1)\) when simulating datasets for
Bayesian t-test. A) ECDF difference plot for the derived quantity
\(f(i, y) = (i - \frac{1}{2})(\text{sd}(y) - 1)\) (model\_sd\_m1). B)
and C) Separate binary prediction calibration plots for datasets with
high/low standard deviation.}

\end{figure}%




\end{document}
